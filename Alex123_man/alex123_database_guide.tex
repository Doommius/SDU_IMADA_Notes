%Template by Mark Jervelund - 2015 - mjerv15@student.sdu.dk

\documentclass[a4paper,10pt,titlepage]{report}

\usepackage[utf8]{inputenc}
\usepackage[T1]{fontenc}
\usepackage {tikz}
\usetikzlibrary {positioning}
\usepackage[english]{babel}
\usepackage{amssymb}
\usepackage{amsmath}
\usepackage{amsthm}
\usepackage{graphicx}
\usepackage{fancyhdr}
\usepackage{lastpage}
\usepackage{listings}
\usepackage{algorithm}
\usepackage{algpseudocode}
\usepackage[document]{ragged2e}
\usepackage[margin=1in]{geometry}
\usepackage{color}
\usepackage{datenumber}
\usepackage{venndiagram}
\usepackage{chngcntr}
\usepackage[utf8]{inputenc}
\usepackage[english]{babel}
\usepackage{amssymb,amsmath,amsthm}
\usepackage{mathtools}
\newtheorem{theorem}{Theorem}
\usepackage{hyperref}
\hypersetup{
    colorlinks=true,
    linkcolor=blue,
    filecolor=magenta,      
    urlcolor=cyan,
}

\usepackage{mathtools} % Bonus
\DeclarePairedDelimiter\norm\lVert\rVert
\setdatetoday
\addtocounter{datenumber}{0} %date for dilierry standard is today
\setdatebynumber{\thedatenumber}
\date{}
\setcounter{secnumdepth}{0}
\pagestyle{fancy}
\fancyhf{}
\title{Alex123 database upgrade guide}

\newcommand{\Z}{\mathbb{Z}}
\lhead{Alex123 database upgrade guide}
\rhead{Mark Jervelund}
\rfoot{Page  \thepage \, of \pageref{LastPage}}
\counterwithin*{equation}{section}

\begin{document}

\section{Introduction}

This report outlines the use of the alex123 database update tools.
\\

I highly recommend doing the update via the server, this speeds up the process by a lot and means the whole process is faster, and the risk of disconnecting and having to restart the process is not there.



The general path for updating the database should be:

\begin{enumerate}
\item Clone repository(Download)
\item Make changes to the data
\item Push the data(Upload the changes)
\item Login to server and update the code.
\end{enumerate}	

\newpage
\section{Inspecting the database via Mysql Workbench}

Inspecting the database can be useful to check if a single value is present, this is useful when debugging data import, and why something might not import. This is most likely not needed but it's a very nice thing to know if you're having issues updating the data.


The workbench can be loaded from the link below.

\url{https://www.mysql.com/products/workbench/}

When the program has been installed you need to add the server, This is done by pressing the plus next to mysql connections at the top left of the window.

\vspace{5mm}

The credentials for the database are:

The server hostname is alex123.sdu.dk

The user is "alex\_123\_admin"

The password is \textcolor{red}{"placeholder"}
\vspace{5mm}

When you've connected to the database you can run queries directly on the database.
\newpage
\section{remote update(w.r.t Server)}

To preform a update of the data on the database at alex123.sdu.dk

First download and install PHP This can be done from the following link.
\\
\vspace{5mm}
\url{http://php.net/downloads.php}

\vspace{5mm}
When the program has been installed the code for the updater can be downloaded from
\vspace{5mm}

\url{https://git.embl.de/ejsing/alex123_online_database}

\vspace{5mm}
It is recommended doing this via the git tool so the changes to the data can be added to the repository. please see the Using Git section

\section{Local Update(w.r.t. server)}

For updating the database via the server you need to connect to the server at SDU. However you'll first need to make the updates to the git repository. and how to this can be found in the git section. but essentially you need to clone to project, make the changes and push the changes to EMBL.

To do this a SSH Client is required. There is one included naively in mac, Linux and the newest releases of windows.

If you're running on a older version of windows, putty can be installed, putty can be found at \url{https://www.putty.org/}
\vspace{5mm}

The Credentials needed to login to the server is your SDU employee/student credentials, note: currently cse and martin should have access.

\begin{lstlisting}
ssh {SDUusername}@alex123.sdu.dk
ssh cse@alex123.sdu.dk
\end{lstlisting}

From here to to the folder with the project, this is done by using the following command

\begin{lstlisting}
cd /var/www/import-scripts/
\end{lstlisting}

From here the code can be updated by running


\begin{lstlisting}
git pull
\end{lstlisting}


from here the update script can be called by running

\begin{lstlisting}
 nohup sh import_ubuntu.sh &
\end{lstlisting}

Now the server will preform the update and station on the update can be see at 

\url{https://alex123.sdu.dk/status.php}

\newpage 
\section{Using git}

Git is a source code management tool used for allowing collaboration and version control over code bases, In this case it's used for storing the code and the data in txt format.

\subsection{Installation}
Git can be downloaded from:

\url{https://git-scm.com/}

I do recommend download gitKraken too as it ads a user interface and makes it easier to work with.

\url{https://www.gitkraken.com/download}


\subsection{GIT}

\subsubsection{Command line}

For using git in the command line there are 4 commands you need to use.
\begin{lstlisting}
Downloads the repository to you local computer at the current folder.
Git clone https://git.embl.de/ejsing/alex123_online_database.git

Downloads the newest code from the repository and updates the current files and merges changes the local and the remote version.
Git pull

Adds changes and new files to the project.
git add *


Commits the changes as a version to the repository
git commit -m "message"


uploads the changes to the online repository.
git push

\end{lstlisting}

\subsubsection{GITKraken}

TODO


\end{document}
