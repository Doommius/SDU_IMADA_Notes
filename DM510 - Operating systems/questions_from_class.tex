%Template by Mark Jervelund - 2015 - mjerv15@student.sdu.dk

\documentclass[a4paper,10pt,titlepage]{report}

\usepackage[utf8]{inputenc}
\usepackage[T1]{fontenc}
\usepackage[english]{babel}
\usepackage{amssymb}
\usepackage{amsmath}
\usepackage{amsthm}
\usepackage{graphicx}
\usepackage{fancyhdr}
\usepackage{lastpage}
\usepackage{listings}
\usepackage{algorithm}
\usepackage{algpseudocode}
\usepackage[document]{ragged2e}
\usepackage[margin=1in]{geometry}
\usepackage{color}
\usepackage{datenumber}
\usepackage{venndiagram}
\usepackage{chngcntr}
\setdatetoday
\addtocounter{datenumber}{0} %date for dilierry standard is today
\setdatebynumber{\thedatenumber}
\date{}
\setcounter{secnumdepth}{0}
\pagestyle{fancy}
\fancyhf{}

\newcommand{\Z}{\mathbb{Z}}
\lhead{Operating systems (DM510))}
\rhead{Mark Jervelund (Mjerv15)}
\rfoot{Page  \thepage \, of \pageref{LastPage}}
\counterwithin*{equation}{section}

\begin{document}
\renewcommand{\thepage}{\roman{page}}% Roman numerals for page counter
\tableofcontents
\newpage
\setcounter{page}{1}
\renewcommand{\thepage}{\arabic{page}}
\section{Course description}
To describe the services an operating system provides to
users, processes, and other systems\\
To discuss the various ways of structuring an operating
system\\
To explain how operating systems are installed and
customized and how they boot\\
\newpage
\section{Question 1 - Operating System Services - ch 2}
\subsection{Questions from lecture}


\textbf{What is a system call.\\}
\hspace{10mm} Programming interface to the services provided by the operating system \\ \vspace{5mm}


\textbf{What groups of system calls are there. \\}
\hspace{10mm}File malipulation, process control, security and protection, communication, I/O device manipulation \& information managetment\\ \vspace{5mm}



\textbf{why would you use a system call API.\\}
\hspace{10mm}Ease of use, Reduce failure, reuseability/portablility \\ \vspace{5mm}


\textbf{What ways are there to pass parameters to the os.\\}
\hspace{10mm} Stack + reference, register, memory -> send reference \\  \vspace{5mm}



\textbf{How can an operateing system be structured(archichture wise)\\}
\hspace{10mm} Layered structure, module based, microkernal \& hybrid systems\\ \vspace{5mm}

 
\textbf{what code is first read when you turn on your computer\\}
\hspace{10mm}Bootstrap code located in EPROM \\ \vspace{5mm}

\newpage
\section{Question 2 - Process Concept and Multithreaded Programming - ch 3 \& 4}

\subsection{Questions from lecture}
\textbf{What state can a process by in?\\}
\hspace{10mm}New.\\
\hspace{10mm}Ready.\\
\hspace{10mm}running.\\
\hspace{10mm}waiting.\\
\hspace{10mm}terminated.\\
\vspace{5mm}

\textbf{What is a process control block?\\}
\hspace{10mm}A block where the OS store information about a program.\\ 
\hspace{10mm}Used for content switching and stores the following information:\\
\hspace{20mm}Process state\\
\hspace{20mm}Process number\\
\hspace{20mm}Process counter\\
\hspace{20mm}Registers\\
\hspace{20mm}Memory limits\\
\hspace{20mm}List of openfiles\\


\vspace{5mm}
\textbf{Describe ways to do IPC\\}
\hspace{10mm}Message parsing and memory shareing. \\
\vspace{5mm}



\textbf{What os the difference betwen a process and a thread\\}
\hspace{10mm}A process is a thing we need to exceute, and thread is a subtask of a process.\\
\vspace{5mm}



\textbf{What advantages is there when using threads?\\}
\hspace{10mm}Threading lets you work on multiple cores at the same time, therefor letting you run a job faster.\\
\vspace{5mm}



\textbf{Differnence between parallelism and concurency?\\}

\hspace{10mm}	Parallelism - same task spread on muliple threads\\
\hspace{10mm}	Concurency - program running with multiple threads\\
\hspace{10mm}  	We can have Concurency without Parallelism but not the other way around\\
\vspace{5mm}



\textbf{What are the most common API's for user level threads\\}
\hspace{10mm} Boost (maybe builds on pthread) \\
\hspace{10mm} Pthread \\
\hspace{10mm} Java thread \\
\hspace{10mm} Winthread \\
\vspace{5mm}




\textbf{What are the implicit threading methods.\\}
\hspace{10mm} Thead pools \\
\hspace{10mm} Openmp \\
\hspace{10mm} Grand Central Dispatch \\

\newpage

\section{Question 3 - Process Scheduling - ch 5}

\subsection{Questions from lecture}
\textbf{When is it Relevent for the scheduler to take  decision.\\}
\hspace{10mm}When a new process queued, and a new processes is started.\\
\hspace{10mm}4 cases\\
\hspace{20mm}when a process running state to waiting.\\
\hspace{20mm}when a process terminates.\\
\hspace{20mm}When a process is queued\\
\hspace{20mm}When a process goes from waiting to ready\\ \vspace{5mm}

\textbf{What is dispatch latency.\\}
\hspace{10mm} time it takes for a interupting process to become active \\
\hspace{10mm} move memory and registers from the running task, move new tasks there and start it \\\vspace{5mm}



\textbf{What scheduling ctiteria can we use.\\}
\hspace{10mm} maximize CPU util, low latency for take critical jobs jobs.\\
\hspace{10mm} runtime, priority, CPU bound, I/p bound \\\vspace{5mm}



\textbf{Describe the scheduling algortirhms: FCFS, SJF, Shortest remaining time fist, RR.\\}
\hspace{10mm} First come first servce\\
\hspace{10mm} Shortest job first \\
\hspace{10mm} Shortest time remaining first \\
\hspace{10mm} Round robin\\
\hspace{10mm}give each process the same amount of time and loop over them until they're done. circular queue\\\vspace{5mm}




\textbf{Describe priority schduling and aging\\}
\hspace{10mm} The longer it has been in the queue the higher the priority the job has\\


\textbf{What is the difference betweeen asymmetric and symmetric multiprocessing\\} \vspace{5mm}

\hspace{10mm} Asymmmetric means tasks wont wait for other things \\
\hspace{10mm} Not all processes has the some capapicities\\
\hspace{10mm} All the kernals can do the same.\\ \vspace{5mm}



\textbf{What is a memory stall\\}
\hspace{10mm} Waiting for memory ? \\\vspace{5mm}


\textbf{What is the difference between soft and hard real-time systems.\\}
\hspace{10mm} Soft -  Don't time life critital systems while hard real time systems have.\\
\vspace{5mm}

\textbf{Describe rate montonic scheduling and earlist deadline scheduling\\}
\hspace{10mm} rate montonic scheduling - Schedule jobs in intervals, like a, b, c.\\
\hspace{10mm} earlist deadline scheduling - Schedule the process with the first deadline first.\\

\newpage
\section{Question 4 - Synchronization - Lecture 5}

\subsection{Questions from lecture}

\textbf{Describe the terms "race condition" and "Critical section"\\}

\hspace{10mm} When two proceses want to access a shared resource. that may be unstable if its done in a \\ \hspace{10mm} non-locked way\\
\hspace{10mm} 
Race condition is when multiple processes/threads are competing about the same resources and \\ \hspace{10mm}  a undesired output may happen. \\ \vspace{5mm}
\hspace{10mm}  Critical section is a section of code that sensitive to race conditions.\\



\textbf{What should a solution to the critical section satisfy \\}
\hspace{10mm}  Mutual exclution, progress, and \\
\hspace{10mm}  that only one thread can be active in the critical section at the time. \\\vspace{5mm}




\textbf{What is preemptive vs. non preemptive \\}

\hspace{10mm}  preemptive is the act is  temporarily interrupting a process\\

\hspace{10mm}  non-preemptive -  When a process enters the state of running, the state of that process is not \\ \hspace{10mm} deleted from the scheduler until it finishes its service time.\\ \vspace{5mm}




\textbf{Describe "test and set" and "compare and swap". \\}

\hspace{10mm}  Two different ways of implementing a mutex lock, that are garenttted to be excuted in an atomic \\ \hspace{10mm} way.\\

\hspace{10mm} Test and set - \\

\hspace{10mm}  Compare and swap can only be used by one function, its checks for a codition and sets swaps \\ \hspace{10mm}  two value if the condition is met.\\  \vspace{5mm}




\textbf{What is a mutex and a semaphore \\}

\hspace{10mm}  a Mutex lock is a lock where only the process holding the lock is allowed to perform actions in \\ \hspace{10mm}  	the locked section. \\
\hspace{10mm}  A mutex contains a boolean and is acquired and released.\\


\hspace{10mm}  A binary semaphore is the same as a mutex. \\

\hspace{10mm}  a counting semaphore is when the lock uses a counter if multiple processes can be active within \\
\hspace{10mm}  the lock. \\ \vspace{5mm}


\textbf{Describe some classic problems of synchronization. \\}

\hspace{10mm}  Readers writers problems.\\
\hspace{10mm} dining philosopher problem. \\ \vspace{5mm}


\textbf{What is a monitor \\}
\hspace{10mm}  We can put functions into the monitor and only one process can be active within the monitor \\
\hspace{10mm} at the time, else the process will have to wait in the queue.


\newpage
\section{Question 5 - Deadlocks - Lecture 6}
\subsection{Questions from lecture}
which 4 conditions most hold for a deadlock\\
\hspace{10mm} and discribe them \\
\hspace{10mm} mutual exclusion\\
\hspace{20mm} Two processes lock each other out due to interupt  \\
\hspace{10mm} hold and wait\\
\hspace{20mm} Processes is waiting for recourses to be released by other process \\
\hspace{10mm} circular wait\\
\hspace{20mm} two or more processes are waiting for each other to release resources \\
\hspace{10mm} no preemption\\
\hspace{20mm} cant remmeber \\


Describe the resource graph\\
\hspace{10mm} Graph the shows the dependency of recources and processes \\


What methods are there to handle deadlocks \\
\hspace{10mm} avoid deadlock \\
\hspace{10mm} allow deadlocks and recover \\
\hspace{10mm} ignore them  \\


What is a safe state \\
\hspace{10mm} A state where deadlocks cant occur \\


Describe the general idea of bankers algoritm \\
\hspace{10mm} each process says how many recources they require, and the schduler assigns the so no deadlocks occur\\


How can a deadlock be deteched \\
\hspace{10mm} multiple processes are waiting for each other, and this can be deteched with a resource graph \\


How can you recover from a deadlock\\
\hspace{10mm} kill processes that are in the deadlock until its resolved. \\

\end{document}
