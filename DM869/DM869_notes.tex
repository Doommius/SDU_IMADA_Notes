%Template by Mark Jervelund - 2015 - mjerv15@student.sdu.dk

\documentclass[a4paper,10pt,titlepage]{report}

\usepackage[utf8]{inputenc}
\usepackage[T1]{fontenc}
\usepackage[english]{babel}
\usepackage{amssymb}
\usepackage{amsmath}
\usepackage{amsthm}
\usepackage{graphicx}
\usepackage{fancyhdr}
\usepackage{lastpage}
\usepackage{listings}
\usepackage{algorithm}
\usepackage{algpseudocode}
\usepackage[document]{ragged2e}
\usepackage[margin=1in]{geometry}
\usepackage{color}
\usepackage{datenumber}
\usepackage{venndiagram}
\usepackage{chngcntr}
\usepackage[utf8]{inputenc}
\usepackage[english]{babel}
\usepackage{amssymb,amsmath,amsthm}
\usepackage{mathtools}
\newtheorem{theorem}{Theorem}

\usepackage{mathtools} % Bonus
\DeclarePairedDelimiter\norm\lVert\rVert
\setdatetoday
\addtocounter{datenumber}{0} %date for dilierry standard is today
\setdatebynumber{\thedatenumber}
\date{}
\setcounter{secnumdepth}{0}
\pagestyle{fancy}
\fancyhf{}
\title{DM556/866 DBMS}

\newcommand{\Z}{\mathbb{Z}}
\lhead{DM869)}
\rhead{Mark Jervelund (Mjerv15)}
\rfoot{Page  \thepage \, of \pageref{LastPage}}
\counterwithin*{equation}{section}

\DeclareUnicodeCharacter{344}{TEST}

\begin{document}
\begin{titlepage}
\centering
    \vspace*{9\baselineskip}
    \huge
    \bfseries
     Advanced Topics in Concurrent Systems \\ DM869 \\
    \normalfont 
    Mark Jervelund  \\
    Mark@jervelund.com\\
    Doommius.com/notes.php 	\\
    \vspace*{9\baselineskip}
    \normalfont
	\includegraphics[scale=1]{SDU_logo}
    \vfill\ 
    \vspace{5mm}
    IMADA \\

    \textbf{\datedate} \\[2\baselineskip]
\end{titlepage}

\renewcommand{\thepage}{\roman{page}}% Roman numerals for page counter
\tableofcontents
\newpage
\setcounter{page}{1}
\renewcommand{\thepage}{\arabic{page}}

\section{notes}

w.r.t With regards to

s.t. such that
\section{Introduction to inference systems}

Rules

\begin{theorem}

$\frac{•}{num(Z)}[Zero]$

\end{theorem}

\begin{theorem}

$\frac{num(Z)}{num(Sx)}[Succ]$

\end{theorem}
\subsubsection{num(Z)}
Num(Z) is derivable iff x encodes a nautral number, if any derivation for number x has exatly height n, then x encodes n
proff by induction, on the structure pf the given deviration for num(x)

Care Zero
The derivation starts with rule[Zero] hence X must be >, the height must be 1

case one
Num(Z) is derivable iff x encodes a nautral number, if any derivation for number x has exatly height n+1, then x encodes n
proff by induction, on the structure pf the given deviration for num(x)

Care Zero
The derivation starts with rule[Zero] hence X must be n+1, the height must be n+1

\subsubsection{succ}
Case Zero

The derivation rulestarts with rule[succ] hence X=Sy for somey, we have a derivation for num(Y), Its height, Say m.

By induction HP, y encodes m-1 thus Sy, encodes m-1


\subsubsection{Add}
\begin{theorem}

$\frac{add(w,x,y)}{add(Sw, X, Sy)}[Succ]$

\end{theorem}

\includegraphics[scale=0.4]{draw.io/fig1.png}

add(W,X,Y) is derivable iff W+X = Y

Prove by induction on the derication of add(W,X,Y)

Case +$\Z$
There is no inductive step

W = Z, X = x = Y, O+X=X$ \check{}$

Case +S W= Sw, X = x, Y = Sy,

We have a serivation for add(w,x,y)
We can apply the inductive hypothesis(ind. HP)
w+x=y, W=1+w, Y=1+y, we conclude that 1+w+x=1+y, W+X =Y


\subsubsection{sub}
\begin{theorem}

$\frac{num(Z)}{num(Sx)}[Succ]$

\end{theorem}

sub(w,x,y) def, rules s.t. sub(q,x,y) is derivable iff w-x=y.

It can be proved that there is no proof for this.



\subsubsection{if then}

\begin{theorem}

$\frac{if x+y=z}{then w-x=y}\frac{add(x,y,w)}{sub(w,x,y)}$

\end{theorem}

\newpage

\section{Conculus of the cumunication system}

C is a channel

C = new channel ("IP eg 10.130.10.42")

C.open(); connect
C.send(42);

x: c.recv()
P:


Def. a labelled transition sstem is (S, L, $\rightarrow$)
\begin{itemize}
\item S is a set of steates (processes)
\item K us a set of lables (Actions)
\item $\rightarrow \subseteq S\times L\times S $ is trasition relation 
\end{itemize}

Natation S $ \xrightarrow[\text{}]{\text{e}} $ S' means (S,e, S')$ \in \rightarrow$ 


$P::= \emptyset$ \hspace{20mm}//Termination program

$\overline{C}.P$ \hspace{24mm}//send on channel c and continue as P 

$\overline{C}.P $\hspace{25mm}// Recieve on channel c and continue as P 


$ \frac{}{C.P \xrightarrow[\text{}]{\text{c}}P}  $ [Send]

$ \frac{}{ \overline{C}.P \xrightarrow[\text{}]{\overline{ \text{c} }}P}  $ [Recieve]

$ \frac{P \xrightarrow[\text{}]{{ \text{c} }}P' \hspace{5mm} Q\xrightarrow[\text{}]{{ \text{a} }}Q'}{P|Q \xrightarrow[\text{}]{{ \text{t} }}P'| Q'}  $ [Com]




How can i see that two programs are running at the same time.

P | Q \hspace{21mm}// P and Q urn concurrently

$c.P | \overline{C}.Q \rightarrow P | Q$ \hspace{2mm} //If we have two nodes, c, and $\overline{c}$ and c wants to send to $\overline{c}$ and $\overline{c}$ want's to recieve from c, this is syncronys transmition. eg, a communication can't fail.




Spec R spec 1 .... Secn Rimpl

R should be an equvalence relation between two processes
	-transitive: reason about chain refinemnt
	Reflexivity PRP
	Symmytry Spec R impl. -> Impl R spec
	
	
Def
A cotntect is any term generated
$
C:= [-] | \alpha .C | C+P | (P|C)
$

$
P::= \alpha . P | O | P+P' | (P|P')
$



Def
A trace is a sequence$  \alpha_1 ....... \alpha_n \in ACT $

(ACT)* is the finite word containing all possible programs




The set of traces of a process P is the set of traces(P) $\{\alpha_1 ..... \alpha_n\}$


\begin{equation}
Traces(User) = \{\in;  \overline{P}; \overline{P}.enter; \overline{P}.enter.exit; .....\} = \{\in\} \cup \{\overline{P}t | \in traces (U_1)\}  
\end{equation}
		     
		     
Def a Relation R is a visimulation if
Whenever two processes PRQ it holds that :
We have two conditions
	if P can proform some transition 

	\begin{equation}
		\text{if } p  \xrightarrow[\text{}]{{ \alpha }} P' \text{ then there is } Q  \xrightarrow[\text{}]{{ \alpha }} Q' \wedge R'RQ'
	\end{equation}
	\begin{equation}
	\text{if }Q  \xrightarrow[\text{}]{{ \alpha }} Q' \text{ then there is } P  \xrightarrow[\text{}]{{ \alpha }} P \wedge ' R'RQ'
	\end{equation}


Def P,Q are called Bisimilar (P ~ Q) iff there is a bisimulation r st. PRQ


Desiderata for behavioval eq.

-Equvilance relation

-Congruence (w.r.t to the syntax for out programming language)

def. A relation R over CCS processes is a bisimulation if whenever $(P,Q) \in R$ it hold that for any lable $\alpha$ :
\begin{theorem}
1) If $p \xrightarrow[\text{}]{{ \alpha}} p'$ then there is q' s.t. $Q \xrightarrow[\text{}]{{ \alpha}} Q'$ and $(P',Q') \in R$
\end{theorem}
\begin{theorem}
2) symmetric of 1)
\end{theorem}





P and Q are called bisimilar (P ~ Q) if there a bisimulation R. S.t. $(P, Q) \in R$


Lemma: Bisimularirt ~ is an equivilance relation

Proof
	-Reflexivity $\forall P$, P ~ P $p \xrightarrow[\text{}]{{ {\alpha} }} p'$ , $p \xrightarrow[\text{}]{{ {\alpha} }} p'$
		-	$I \subseteq ~$
		-Prove that I is a bisimulartion: $(P,Q) \in I$ => P=Q => $p \xrightarrow[\text{}]{{ {\alpha} }} p'$ then $Q \xrightarrow[\text{}]{{ {\alpha} }} p'$  $(P',P') \in I $
		
	-Symmetry:
		P~Q => Q~P.
			- By def. of bisimilairty there is a bisimulation R such that $(P,Q) \in R$
				Claim $R^-1$ (The relation where you flip every pair) = $\{(x,y)|(y,z) \in R\}$ is a bisimulatin
					$(Q,P) \in R^-1$ is bisimular, You can conclude Q~P
					

Given that R is a bisimulation then $R^{-1}$ is a bisimulation for any R.
let $(s,t) \in R^{-1}$. 
1) For any $S \xrightarrow[\text{}]{{ {\alpha} }} S'$,let T' be any s.t. $(S',T') \in R $(There is at least one since R is a Bisiulation).
	(Since S is in relation with T by our definition, by this point by construction.)
	It follows that $T \xrightarrow[\text{}]{{ {\alpha} }} T'$, $(S',T') \in R^{-1}$ by The above formula. this proves the first conditioning of the def. of bisumulation.

2.) Likewise: given $T \xrightarrow[\text{}]{{ {\alpha} }} T'$, there is a transition $S \xrightarrow[\text{}]{{ {\alpha} }} S'$ s.t. $(T', S') \in R$ hence $(T', S') \in R^{-1}$

-Transitivity: P~Q, Q~S => P~S, 
	\textbf{The strategy here is to show that it is similar to what we did before.}
	-From P~Q it follows that $\exists R_1$ s.t. $R_1$ is a bisumulation and $(P,Q) \in R_1$
	-From Q~S it follows that $\exists R_2$ s.t.$ R_1$ is a bisumulation and $(Q,S) \in R_2$

	Define $R = R_1 \dot R_2 = \{(x,<) | \exists y. (x,y) \in R_1 \wedge (y,z) \in R_2 \}$
	If R is a bisimulation by constrction ($P,S) \in R$ hence P~S
	


$(T_1, T_2) \in R$
\textbf{If t1 can preform an action, then t2 can do the same with the same lables.}
1) let $T_1 \xrightarrow[\text{}]{{ {\alpha} }} T_2$, by construction of R. $T_3$ s.t. 
($T_1, T_3) \in R_1$, therefor $T'_3$ s.t. $T_3 \xrightarrow[\text{}]{{ {\alpha} }} T_3' \wedge (T'_1, T'_3) \in R_1$
$(T_3, T_2) \in R_2$, since $R_2$ is a bimiulation, $T_2 \xrightarrow[\text{}]{{ {\alpha} }} T_2'$ s.t. $(T'_3, T'_2) \in R$
\textbf{Inset picture here}

$T'_1 R_1 T'_3 R_2 T'_2 => T'_1 R T'_2$

2) $T_2 \xrightarrow[\text{}]{{ \alpha }} T_2', $

\underline{Lemma:} ~Is a bisimulartion
	See notes
	
\underline{Lemma:} ~ is a congruence
	$\forall C, P~Q => C[P] ~C[Q]$
	The result is a result for a weaker/eqivilant one.
	Proof: by structual induction
		Let P~Q, Proceed by strucurtal induction on C,
		\underline{Base:} C~C, (No holes) It follow by reflexivity of ~
			if C= [\_] then C[P] = P~Q = C[Q]
		\underline{induction case:}
			\begin{itemize}
			\item $C = \alpha.[_] \alpha.P ~ \alpha.Q$
			\item $C = S|[_] S|P ~S|Q$
			\item $C = [_]|S S|P ~S|Q$
			\item $C = S+[_]$
			\item $C = [_]+S$
			\item $C = [_]\textbackslash L$
			\end{itemize}

	Lemma: If P~Q then:
	1) a.P ~ a.Q
	2) $\forall S S|P ~S|Q$
	3) $\forall S+P ~ S+Q$
	
1) P~Q => a.P ~A.Q let R be a Bisimulation s.t. $(P,Q) \in R$. Def. $R' = R \cup \{ (a.P, a,Q) \}$
	The two conditions
	\begin{itemize}
	\item $a.P \xrightarrow[\text{}]{{ {\alpha} }} p$ the other process can only reply with $a.Q \wedge a.Q\xrightarrow[\text{}]{{ {\alpha} }} Q$. $(P,Q) \in R'$, 
	\end{itemize}
2)$P~Q => \forall S, S|P | S|Q. $Let R be a bisimulation $(P,Q) \in R. $
	-Def. R as the following relation$ R = \{ (S|P,S|Q) | P~Q,$ P,Q,S are CCS Processes \}
		Assume that $S|P \xrightarrow[\text{}]{{ {\alpha} }} S'|P'$. Derivations for $S|P \xrightarrow[\text{}]{{ {\alpha} }} S'|P'$ have thererfo: the first rule applied is 1) Lpar, 2) Rpar, 3, com

1) Lpar
	\begin{equation}
	\frac{S\xrightarrow[\text{}]{{ {\alpha} }} S'}{S|P \xrightarrow[\text{}]{{ {\alpha} }} S'|P} (P = P') \frac{S\xrightarrow[\text{}]{{ {\alpha} }} S'}{S|q \xrightarrow[\text{}]{{ {\alpha} }} S'|Q}
\end{equation}	 

$(S'|P, S'|Q) \in R$ by def of R

2) Rpar
	\begin{equation}
	\frac{P\xrightarrow[\text{}]{{ {\alpha} }} P'}{S|P \xrightarrow[\text{}]{{ {\alpha} }} S|P'} (S = S') \text{ Since } P~Q \exists {Q\xrightarrow[\text{(*)}]{{ {\alpha} }} Q'} P' \underset{~}{\dag}Q'
\end{equation}

\begin{equation}
\text{From}(\star) \frac{Q\xrightarrow[\text{}]{{ {\alpha} }} Q'}{S|Q \xrightarrow[\text{}]{{ {\alpha} }} S|Q'} \text{And from (\dag)} (S|P̈́', S|Q') \in R
\end{equation}	
3) com
\begin{equation}
\frac{S\xrightarrow[\text{}]{{ {\alpha} }} S'\text{ } P\xrightarrow[\text{}]{{ \overline{\alpha} }} P'}{S|P \xrightarrow[\text{}]{{ {\tau} }} S'|P'} \text{ Since } P ~ Q, Q\xrightarrow[(\star)]{{ \overline{\alpha} }} Q' \text{  } P' \underset{(\dag)}{~} Q'
\end{equation}	

\begin{equation}
\text{From }(\star)  \frac{S\xrightarrow[\text{}]{{ {\alpha} }} S'\text{ } Q\xrightarrow[\text{}]{{ \overline{\alpha} }} Q'}{S|Q \xrightarrow[\text{}]{{ {\tau} }} S|Q'} \text{ And from } (\dag) (S'|P',S'|Q') \in R.
\end{equation}	

Likewise $S|Q \xrightarrow[\text{}]{{ {\alpha} }}  S'|Q'$.... 



PFor S+P ~ S+Q use the relation
\textbf{Picture from phone}



P|Q ~Q|P prove that R = \{ (P|Q, Q|P) \} is a bisimular


For an LTS (S, ACT, ->) It's saturation is the LTS S, ACT, =>) where => is the least relation s.t.

We can ignore internail computation
\begin{equation}
\frac{•}{S \rightarrow{[\text{}]{{ {\tau} }} S}} \frac{S_1 \rightarrow{[\text{}]{{ {\tau} }} S'_1 \text{  }S'_1 \xrightarrow{[\text{}]{{ {\alpha} }} S'_2}\text{  }S'_2 \xrightarrow{[\text{}]{{ {\alpha} }} S_2}}{\text{  }S_1 \rightarrow{[\text{}]{{ {\alpha} }} S_2}
\end{equation}

\end{document}