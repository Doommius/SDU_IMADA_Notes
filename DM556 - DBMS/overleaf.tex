%Template by Mark Jervelund - 2015 - mjerv15@student.sdu.dk

\documentclass[a4paper,10pt,titlepage]{report}

\usepackage[utf8]{inputenc}
\usepackage[T1]{fontenc}
\usepackage[english]{babel}
\usepackage{amssymb}
\usepackage{amsmath}
\usepackage{amsthm}
\usepackage{graphicx}
\usepackage{fancyhdr}
\usepackage{lastpage}
\usepackage{listings}
\usepackage{algorithm}
\usepackage{algpseudocode}
\usepackage[document]{ragged2e}
\usepackage[margin=1in]{geometry}
\usepackage{color}
\usepackage{datenumber}
\usepackage{venndiagram}
\usepackage{chngcntr}
\setdatetoday
\addtocounter{datenumber}{0} %date for dilierry standard is today
\setdatebynumber{\thedatenumber}
\date{}
\setcounter{secnumdepth}{0}
\pagestyle{fancy}
\fancyhf{}
\title{DM556/866 DBMS}

\newcommand{\Z}{\mathbb{Z}}
\lhead{DBMS (DM556) 2018)}
\rhead{Mark Jervelund (Mjerv15)}
\rfoot{Page  \thepage \, of \pageref{LastPage}}
\counterwithin*{equation}{section}

\begin{document}
\renewcommand{\thepage}{\roman{page}}% Roman numerals for page counter
\tableofcontents
\newpage
\setcounter{page}{1}
\renewcommand{\thepage}{\arabic{page}}
\section{Latex guide}
$
Set \ A = \emptyset 
$
\\
$
A \mapsto B
$
\\
$
\mid
$


\section{Course description}


\section{Questions from BB}
Review questions:

1. What is a clustered index? How many clustered or unclustered indexes can you build on a table? (Section 8.2.1)

2. What issues are considered in using clustered indexes? What is an index only evaluation method? (Section 8.5.2)

1. Explain the general external merge sort algorithm. Discuss the length of initial runs, how memory is utilized in subsequent merging passes and the cost of the algorithm.
2. Discuss the use of replacement sort to increase the average length of initial runs. How does this affect the cost of external sorting? 


\section{Questions from Lecture}
which xact properties does crash recovery enforce\\
\hspace{10mm}	isolated.\\
\hspace{10mm}	dualbility\\

what is the force/no-force policy (buffer management)?  \\
\hspace{10mm}	when we commit we force the page to be flushed to the disk. \\

what is the steal/no-steal policy (buffer management)\\
\hspace{10mm}	no-steal \\
		

When should a log record be flushed to the disk (write-ahead logging) \\
\hspace{10mm}	if we're waiting for something to be flushed, so we can undo \\
\hspace{10mm}	When we commit, so we can redo \\

why do we need checkpoints in the logging system? \\
\hspace{10mm}	So we don't have to check the entire log. we therefor make checkpoints when we commit, and things are flushed to the disk. \\

what should be done during checkpointing \\
	






	

\section{Currency control part 1 - Lecture 9.}
Book chapture 16 \& 17
\\
The problem is that a DBMS supports multiple users, and arbitrary interleaving commands can lead to an inconsistent result.\\

\begin{lstlisting}[frame=single]
user 1 : loads the value from account into a. | a = 1000
user 1 : a = a -100.                          | a = 1000 - 1000 = 900
user 1 - wait
user 2 : loads the value from account into b. | b = 1000
user 2 : b = b + 100                          | b = 1000 + 100 = 1100
user 2 : stores value b in account            | account = 1100
user 2 done
user 1 : stores value a in account            | account = 900
\end{lstlisting}
so the added value is lost.


\subsection{Formal definition of transaction properties}
ACID\\
A - Atomicity: Either all actions in the xact happen, or none of them happen. \\
C - Consistency - If each Xact is consistent, and the DB starts consistent, It ends up consistent. \\
\hspace{10mm} Its the programmers task to ensure that Xact is consistent. \\
I - Isolation - Execution of one Xact is isolated from that of the other Xacts.\\
D -	Durability - If a Xact commits, its effects persist.\\
\hspace{10mm} If a command has been comitted it 

\subsection{Atomicity and Durability of transaction.}
There are only 2 outcomes of a command, either it commits or it aborts(fails)\\

To ensure this the DBMS uses logging to log all actions and can rollback, or rollforeward to fix any errors that may occur.

\subsection{Isolation}
User can submit Xacts. but each Xacts excecutes as if they were running by them selves. \\
The simplest approach is to run them all serially\\
\hspace{10mm} Low throughput, (missed overlapping of CPU and IO)

It is possible to interleave actions by ordering them useing a serializable schduler to make a sequence.\\


\subsection{Timestamp based}



\section{distributed databases}




External Sorting 



%%%%%%%%%%%%%%%%%%%%%%%%%%%%%%%%%%%%%%%%%%%%%%%%%%%%%%%%%%%%%%%%%%%%%%%%%%%%%%5

\newpage

\section{Questions from lectures 2018}

\subsection{Lecture 1}


\subsection{Lecture 2}
What is a run what is a pass (when discussing external sorting)\\

- What is the difference between a clustered and an unclustered index?\\
 -- Clustered indexes are when the elements itself are also sorted. Unclustered indexes are sorted indexes on unsorted data.\\
 
 - What cost model are we considering? I.e. what are we optimizing for and why?\\
 -- I/O cost, it takes so long time to read and write data.\\
 
 - What are the elements of access time\\
 -- access time = seek time + rotational delay + delay time\\
 
 - What is the task of a buffer manager\\
 -- Read data into memory for processing + backing to disk for persistency\\
 
 - And how does it process a page request\\
 -- slides-01.pdf, page 67\\
 
 - Describe the clock replacement policy\\
 -- First time read, page is set to 0. Next time, if still 0, replace that page. If 1, set to 0. Pin count also has to be 0.

\subsection{Lecture 3}

\subsection{Lecture 4}

- Which is more powerful SQL or realtion algrebra\\
 -- SQL, because it has more operations eg. GROUP BY\\
 
 - Describe the role of the query optimizer\\
 -- Checking if the same operations can be optimized to reduce the number of operations and thus speeding up the request.\\
 
 - What is a reduction factor?\\
 
 - What plan are a syste, r optimizer considering and why?\\
 
 - Describe in short, how dynamic programming is used to enumerate plans\\
 
 - How are nested queries handled?\\

\subsection{Lecture 5}

- Explain the ACID properties.\\
 -- A: Atomicity, "all or nothing" if one transaction fails, the entire transaction fails. \\
 -- C: Consistency, Any transaction will bring the database from one consistent state to another. \\
 -- I: Isolation, All transactions are isolated, which means that they cannot overlap with other transactions. \\
 -- D: Durability,The durability property ensures that once a transaction has been committed, it will remain so, even in the event of power loss, crashes, or errors.\\
 
 - What is a serializable schedule?\\
 -- Same effect as a serial schedule. Guarantees I, isolation.\\
 
 - What is Conflict-serializability?\\
 -- \\
 
 - What is 2PL and Strict 2PL, and what do they guarantee?\\
 
 - Explain phantoms and unrepeatable reads.\\
 
 - Describe different lock methods.\\
 
 - What is index locking?\\

\subsection{Lecture 6}

What information is added to objects in timestamp based approaches?
Read time stamp(RTS), Write time-stamp(WTS) and a commit bit(C)

What is the basic idea in multi version timestamp CC
Let writer write a new version while readers use an appropriate old copy.

What is the Thomas write rule
We can disregard rwites that have never been seen.


What are the phases in the kung-robinson model
Read - Validate - Write

Under what circumstances in optimistic CC a good choice
Read-dominated workloads + system that has lots of extra resources.

What is the most predominant concurrency control system
Locking.


\subsection{Lecture 7 NOSQL-1}

\subsubsection{what are the types of noSQL databases }
Graph, document, key and column


\subsubsection{What issues with relational databases does nosql databases try to solve.}
Scalability, performance.

\subsubsection{What is the CAP theorem}
Consistency, availability, and partition tolerance, 

\subsubsection{What is BASE}
\textcolor{red}{Google this}
B Basically
A Availability,
S Stability, 
E eventual consistency,
\subsubsection{Explain why transaction need in the NOSQL might be less that in RDBMS}


\subsection{Lecture 8 - NO-SQL 2}
\begin{itemize}
\item What is the recommended way to model data in a column based database system\\
\ \ \ \ \ Model it after the queries that you plan on running on it.


\item What does it mean that a database is "eventually consistent"
\ \ \ \ \ When we have multiple nodes. it may take a while for all nodes to get all information. 

\item What are the design choices that make the Cassandra optimized for fast writes\\
\ \ \ \ \ Appending new information instead of updating old values, this is done via timestamps,

\item What partitioning strategy is used in Cassandra/column based databases \\
\ \ \ \ \ Hashing that points to nodes, this allows the program to distribute data.


\item What is a graph database used/designed for \\
\ \ \ \ \ transitive relations, relations between data.

\item How has the tutorial sessions been that last two weeks \\
\ \ \ \ \ Everything is broken on windows.

\end{itemize}


\subsection{Lecture 9 - Crash recovery}
	Which of the ACID properties do the recovery manager hep maintain?
    
    Describe the force and no steal policies and describe which combination we would prefer to use
    
    Describe the WAL protocol
    What are the phases in ARIES
    What is the content of a log record
    
    What is a CLR(Compensation log record)
    
    
\subsection{Lecture 11 - parallel}

	What is the goal of a parallel databases.
    
    How can parallelization be used in databases
    
    Why is shared nothing architecture databases attractive.
    
   	Name 3 partition methods
    
    Describe fragmentation is distributed databases
    
    Explain bloom join
    
    
    
    
    %%%%%%%%%%%%%%%%%%%%%%%%%%%%%%%%%%%%%%%%5
\newpage    
\section{Exam Topics}

Storage and Indexing

Relational Operator Evaluation

Query Optimization

External Sorting

Concurrency Control with Locking

Concurrency Control without Locking

Crash Recovery

Database Tuning

Distributed Query Processing

Distributed Transactions









\section{Key Words}


\subsection{External Sorting}
We first sort the data in n sized chunks. read 
\subsubsection{one way merge sort}
\subsubsection{External merge sort}
\subsubsection{Optimization of general external merge sort}
\subsubsection{B+ Tree}
\subsubsection{Clustered index}
\subsubsection{Unclustered index}


\newpage
\subsection{Relational operator evaluation}


\subsubsection{System catalog (breifly)}
     * Minimum information (system-wide, table, etc.)
     * Other (Cardinality, Size, etc.)
     * Catalog storage in DBMS
\subsubsection{Access paths (~)}
     * Index
     * Iteration
     * Partitioning
\subsubsection{Access Path Selectivity}
 \subsubsection{Select (important to explain IO cost)}
     * Index / No index
     * Sorted / unsorted
     * Cost(s) (Only most efficient implementations)
\subsubsection{Join (important to explain IO cost)}
     * Nested loops (and improvements)
     * Partitioning (Sort-merge, Hash)
     * Cost(s) (Only most efficient implementations)
\subsubsection{Project (important to explain IO cost)}
     * Same as with select and join
\subsubsection{Set operators}
     * Cross product / Intersection as join
     * Union -- duplicate removal
     
     
     
     
\subsection{Query Optimization}

\subsubsection{What is query optimization and why?}
\subsubsection{Relational Algebra Equivalences.}
\subsubsection{Query optimization techniques.}
	* Two main query optimization techniques (1. Heuristic reordering of the relational algebra operations. 2. Systematic estimating the cost of the different execution plans, and choosing one with the lowest cost.) 
\subsubsection{Nested Loop Join, Sort-Merge Join, Hash Join}


\newpage
\subsection{Storage and Indexing}

\subsubsection{Storage}
    * Data on External Storage (Important points to keep in mind: Disks, tapes and records)
    * Physical characteristics of disks and tapes, and how do they affect the design of database systems.
    * Levels of Redundancy (RAID levels)
\subsubsection{Indexes}
    * B+ Tree Indexes
    * Hash-Based Indexes
    * Clustered vs. Unclustered Index
    * Choice of Indexes
    
    
    
    \newpage
\subsection{Concurrency control without locking}

\subsubsection{Short introduction}
- Why concurrency control?
- Advantages of a lock-free scheme
- (Optimistic CC)

\subsubsection{Object based TimeStamp CC}
- Thomas Write Rule
- Recoverability (?)

\subsubsection{Multiversion TimeStamp CC}
- Read\_old\_versions
- Write\_new\_copy

\subsubsection{Kung-Robinson}
- 3 phases: READ; VALIDATE; WRITE
- Validation 3 tests \& serial validation


8 - Concurrency control with locking

Short introduction
- Goal of database tuning

Index selection
- Go through guidelines

Query Tuning
- The idea
- How to speed up important queries









\end{document}

