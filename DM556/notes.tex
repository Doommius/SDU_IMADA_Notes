%Template by Mark Jervelund - 2015 - mjerv15@student.sdu.dk

\documentclass[a4paper,10pt,titlepage]{report}

\usepackage[utf8]{inputenc}
\usepackage[T1]{fontenc}
\usepackage[english]{babel}
\usepackage{amssymb}
\usepackage{amsmath}
\usepackage{amsthm}
\usepackage{graphicx}
\usepackage{fancyhdr}
\usepackage{lastpage}
\usepackage{listings}
\usepackage{algorithm}
\usepackage{algpseudocode}
\usepackage[document]{ragged2e}
\usepackage[margin=1in]{geometry}
\usepackage{color}
\usepackage{datenumber}
\usepackage{venndiagram}
\usepackage{chngcntr}
\setdatetoday
\addtocounter{datenumber}{0} %date for dilierry standard is today
\setdatebynumber{\thedatenumber}
\date{}
\setcounter{secnumdepth}{0}
\pagestyle{fancy}
\fancyhf{}

\newcommand{\Z}{\mathbb{Z}}
\lhead{DBMS (DM556))}
\rhead{Mark Jervelund (Mjerv15)}
\rfoot{Page  \thepage \, of \pageref{LastPage}}
\counterwithin*{equation}{section}

\begin{document}
\renewcommand{\thepage}{\roman{page}}% Roman numerals for page counter
\tableofcontents
\newpage
\setcounter{page}{1}
\renewcommand{\thepage}{\arabic{page}}
\section{Latex guide}
$
Set \ A = \emptyset 
$
\\
$
A \mapsto B
$
\\
$
\mid
$


\section{Course description}


\section{Questions from BB}
Review questions:

1. What is a clustered index? How many clustered or unclustered indexes can you build on a table? (Section 8.2.1)

2. What issues are considered in using clustered indexes? What is an index only evaluation method? (Section 8.5.2)

1. Explain the general external merge sort algorithm. Discuss the length of initial runs, how memory is utilized in subsequent merging passes and the cost of the algorithm.
2. Discuss the use of replacement sort to increase the average length of initial runs. How does this affect the cost of external sorting? 


\section{Questions from Lecture}
which xact properties does crash recovery enforce\\
\hspace{10mm}	isolated.\\
\hspace{10mm}	dualbility\\

what is the force/no-force policy (buffer management)?  \\
\hspace{10mm}	when we commit we force the page to be flushed to the disk. \\

what is the steal/no-steal policy (buffer management)\\
\hspace{10mm}	no-steal \\
		

When should a log record be flushed to the disk (write-ahead logging) \\
\hspace{10mm}	if we're waiting for something to be flushed, so we can undo \\
\hspace{10mm}	When we commit, so we can redo \\

why do we need checkpoints in the logging system? \\
\hspace{10mm}	So we don't have to check the entire log. we therefor make checkpoints when we commit, and things are flushed to the disk. \\

what should be done during checkpointing \\
	






	

\section{Currency control part 1 - Lecture 9.}
Book chapture 16 \& 17
\\
The problem is that a DBMS supports multiple users, and arbitrary interleaving commands can lead to an inconsistent result.\\

\begin{lstlisting}[frame=single]
user 1 : loads the value from account into a. | a = 1000
user 1 : a = a -100.                          | a = 1000 - 1000 = 900
user 1 - wait
user 2 : loads the value from account into b. | b = 1000
user 2 : b = b + 100                          | b = 1000 + 100 = 1100
user 2 : stores value b in account            | account = 1100
user 2 done
user 1 : stores value a in account            | account = 900
\end{lstlisting}
so the added value is lost.


\subsection{Formal definition of transaction properties}
ACID\\
A - Atomicity: Either all actions in the xact happen, or none of them happen. \\
C - Consistency - If each Xact is consistent, and the DB starts consistent, It ends up consistent. \\
\hspace{10mm} Its the programmers task to ensure that Xact is consistent. \\
I - Isolation - Execution of one Xact is isolated from that of the other Xacts.\\
D -	Durability - If a Xact commits, its effects persist.\\
\hspace{10mm} If a command has been comitted it 

\subsection{Atomicity and Durability of transaction.}
There are only 2 outcomes of a command, either it commits or it aborts(fails)\\

To ensure this the DBMS uses logging to log all actions and can rollback, or rollforeward to fix any errors that may occur.

\subsection{Isolation}
User can submit Xacts. but each Xacts excecutes as if they were running by them selves. \\
The simplest approach is to run them all serially\\
\hspace{10mm} Low throughput, (missed overlapping of CPU and IO)

It is possible to interleave actions by ordering them useing a serializable schduler to make a sequence.\\


\subsection{Timestamp based}



\section{distributed databases}




External Sorting 





\section{Notes 2018}

\subsection{Lecture 2}
What is a run what is a pass (when discussing external sorting)



\subsection{Lecture 7}

What are the types of noSQL databases


What isues with relational databases does nosql databases try to solve.

What is the CAP therom

What is BASE

Expain why trasaction need in the nosql might be less that in RDBMS

	
\section{•}	



\end{document}
