%Template by Mark Jervelund - 2015 - mjerv15@student.sdu.dk

\documentclass[a4paper,10pt,titlepage]{report}

\usepackage[utf8]{inputenc}
\usepackage[T1]{fontenc}
\usepackage[english]{babel}
\usepackage{amssymb}
\usepackage{amsmath}
\usepackage{amsthm}
\usepackage{graphicx}
\usepackage{fancyhdr}
\usepackage{lastpage}
\usepackage{listings}
\usepackage{algorithm}
\usepackage{algpseudocode}
\usepackage{MnSymbol}
\usepackage[document]{ragged2e}
\usepackage[margin=1in]{geometry}
\usepackage{color}
\usepackage{datenumber}
\usepackage{venndiagram}
\usepackage{chngcntr}
\usepackage{enumitem}
\setdatetoday
\addtocounter{datenumber}{0} %date for dilierry standard is today
\setdatebynumber{\thedatenumber}
\date{}
\setcounter{secnumdepth}{0}
\pagestyle{fancy}
\fancyhf{}

\newcommand{\Z}{\mathbb{Z}}
\lhead{Computer Architecture (DM548))}
\rhead{Mark Jervelund (Mjerv15)}
\rfoot{Page  \thepage \, of \pageref{LastPage}}
\counterwithin*{equation}{section}

\begin{document}
\renewcommand{\thepage}{\roman{page}}% Roman numerals for page counter
\tableofcontents
\newpage
\setcounter{page}{1}
\renewcommand{\thepage}{\arabic{page}}
\section{Course description}
To describe the services an operating system provides to
users, processes, and other systems\\
To discuss the various ways of structuring an operating
system\\
To explain how operating systems are installed and
customized and how they boot\\
\newpage

%  			\binom{n}{k}
%  \begin{enumerate}[label=(\alph*)]    \end{enumerate}

\section{TA 2}

\subsection{6.3}
\subsubsection{10}
$ \sum\limits^n_{i=1}\binom{10}{i},i \% 2 \neq  	 0 $
\subsubsection{12}
\begin{enumerate}[label=(\alph*)]
\item $2^{10}$\\
\item $C \binom{10}{2}$
\item $1 + \sum\limits^3_{i=1} C\binom{10}{i}$
\item $ C\binom{10}{5} $

\end{enumerate}
\subsubsection{20}
$ |A| = 7, |B| = 9 $\\

$|\{C \subseteq A \cupdot B ||C|=5 \} =  C\binom{16}{5} = 4368 $
\\
\begin{enumerate}[label=(\alph*)]
\item $ C\binom{16}{5} - C\binom{9}{5} = 4242 $
\item $ C\binom{16}{5} - C\binom{9}{5} - {C\binom{7}{5}} $
\end{enumerate}


\subsubsection{24}
Vandermonde's Identity \\

$ \binom{m-n}{k} = \sum\limits_{j=0}^k \binom{m}{j}\binom{n}{k-j} $
\begin{enumerate}[label=(\alph*)]

\item $ \sum\limits_{j=1}^5 \binom{A}{j}\binom{B}{5-j} = \sum\limits_{j=0}^5 \binom{A}{j}\binom{B}{5-j}-\binom{A}{O}\binom{B}{5}$  \\
$\binom{A+B}{5} - \binom{B}{5}  $\\ $ C \binom{16}{5} - C\binom{9}{5} $
\item $\sum\limits_{j=1}^{k-1} \binom{A}{j}\binom{B}{k-j} = \binom{A+B}{5} - \binom{A}{0}\binom{B}{5}-\binom{A}{5}\binom{B}{0} $


\end{enumerate}
\subsection{6.4}
\subsubsection{4}
$ (W+Z)^n - \binom{n}{k} $
\begin{enumerate}[label=(\alph*)]
\item $\binom{200}{99}*2^101*(-3)^99 $

\end{enumerate}
\subsubsection{8}
\begin{enumerate}[label=(\alph*)]
\item forgot

\end{enumerate}
\subsubsection{12}
$ \binom{n-1}{k-1} \binom{n}{k+1} \binom{n+1}{k} = \binom{n-1}{k} \binom{n}{k-1} \binom{n+1}{k+1} $
\subsubsection{14}

\subsubsection{16}

\subsubsection{20}

\subsubsection{22}


\subsection{TA 4 - 7.3}

\subsubsection{2}



\subsubsection{4}

\subsubsection{8}

\subsubsection{10}

\subsubsection{Exam 2015-2}
a \\
Pick 5 elements from the set of numbers \{1,2,3,4,5,6,7,8\}, pick the numbers so at least one pair of numbers x+y = 9 \\
Prove that $\ni x,y \in B$ st. x+y = 9\\

b \\
k+1 elements - A = \{1,...,2k\} \\
$\ni x,y \in B$ st. x,y = 2k+1\\

solution:\\
write $(xi, yi) for (i, 2k+1-i)$ where $i \in {1,2,...k}$ \\
There are K pairs and$ A=\{1,..,2k \} = \{x_i| 1 \leq i \leq k \} \cup \{y_i| 1 \leq i \leq k \} $\\
$x_i+y_i = 2k+1$ \\
"k Boxes" k+1 items by PHP that is a box with 2 items\\

\subsubsection{Exam 2015-2}
Particle 1-dim position at time t will be written as f(t), f(0) = 0, \\

Every second the particle can move forward, or not move at all.\\
The possibility that the particle has moved is \\
\begin{equation}
P(f(t+1)=f(t)) = q
\end{equation}
\begin{equation}
P(f(t+1)=f(t)+1) = 1-q
\end{equation}
write $P_n(r)=P(f(n)=r)$ prob of being in position r at time n \\

\vspace{5mm}
a) \\
Prove that $P_n(r) = C(n,r)\cdot p^r\cdot q^{n-r} $
\\
\begin{enumerate}
\item $P_0(0) = q = \binom{0}{0} \cdot p^0\cdot q^0$ \\
\item $P_0(1) = 0 = \binom{0}{1} $
\item $P_1(0) = q = \binom{1}{0} \cdot p^0\cdot q^1$ \\
\item $P_1(1) = p = \binom{1}{1} \cdot p^1\cdot q^0$ \\
\end{enumerate}

a) \\ 
Algebraic approach\\
Prove by induction over the number of time units t\\
Base Case:
\begin{equation}
n = 0, -> see a
\end{equation}
ind step: t=n+1
\begin{equation}
P_{n+1}(r+1) = P_n(r)*p +P_n(r+1)*q
\end{equation}

\begin{equation}
P_{n+1}(r+1) = \binom{n}{r}\cdot p^r \cdot q^{n-r}\cdot p +\binom{n}{r+1}\cdot p^{r+1} \cdot q^{n-r-2}\cdot p =p^{r+1}\cdot q^{n-r} [\binom{n}{r}+\binom{n}{r+1}] = \binom{n+1}{r+1} \cdot p^{r+1}\cdot q^{n-r}
\end{equation}

\begin{equation}
 = \binom{n}{0} \cdot p^0 \cdot q^n \cdot q = \binom{n+1}{0} P^0 \cdot q^{n+1}
\end{equation}
\vspace{5mm}


b) \\
2 dim space, f(0) = (0,0)

\begin{equation}
P(f(t+1)=f(t)+\binom{1}{0} ) = p
\end{equation}
\begin{equation}
P(f(t+1)=f(t)+\binom{1}{0}) = q = 1-p
\end{equation}

Write $Q_n(r,s) $ for the prob. of being in position (r,s) at time n \\

Claim is that\\
\[
    Q_n(r,s)=\left\{
                \begin{array}{ll}
                  P_n(r) if n = r+s\\
                  Otherwise 0\\
                \end{array}
              \right.
  \]
  
  test
  
  \[
    P(f(t+1)=f(t)+=\left\{
                \begin{array}{ll}
                  \binom{0}{0} ) = 0\\
                  \binom{1}{1} ) = 0\\
                  \binom{-1}{0})= 0 \\
                  \binom{0}{-1})= 0 \\
                \end{array}
              \right.
  \]
  
  

c) \\




\section{test}
\begin{equation}
    x = \begin{cases}
                  r = \frac{3}{10}\\
                  e = \frac{3}{10}\\
                  c = \frac{2}{10}\\
                  u = \frac{1}{10}\\
                  n = \frac{1}{10}\\
        \end{cases}
     y =  \begin{cases}
                  r = \frac{1}{8}\\
                  e = \frac{1}{8}\\
                  l = \frac{1}{8}\\
                  a = \frac{1}{8}\\
                  t = \frac{1}{8}\\
                  i = \frac{1}{8}\\
                  o = \frac{1}{8}\\
                  n = \frac{1}{8}\\
            \end{cases}
\end{equation}  
  

\end{document}
